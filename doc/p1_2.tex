\subsubsection{Key Rotation}
\label{subsubsec:keyrotation}
At any point later in the meeting, the leader can generate a new 32-byte value $\mk'$. The leader
performs steps \ref{participantJoinRekeyStart}-\ref{participantJoinRekeyEnd} of ``Participant Join
(Leader)'' for all participants, with the updated $\mk'$ value. All participants see the rekey
signal on their signaling channel, and perform step \ref{participantJoinRekeyNonLeader} of
``Participant Join (Non-Leader)''. Participants should not immediately encrypt using the new meeting
key; they should wait about 2 seconds, to ensure all participants smoothly transition over. Each
participant ensures $\mkSequenceNumber$ in the ciphertext sent by the leader is greater than the
previously known $\mkSequenceNumber$; otherwise the key rotation is ignored.

The leader should trigger a rekey whenever a participant enters or leaves the meeting, but not more
than every 15 seconds (to prevent thrashing). As an important security measure, users joining the
meeting never get to see keys that are more than 15 seconds old, since otherwise they could view
encrypted in-meeting chats deep into the meeting's past. Similarly, users who leave a meeting might
be able to decrypt meeting content sent in the next 15 seconds (or up to a couple of minutes if the
server is suppressing messages to prevent the leader from rotating the key). But likewise this
rekeying strategy shouldn't delay a user from joining a meeting.

It's important to note each $\mk$ is independently generated, so knowing the previous $\mk$ provides no information about the subsequent $\mk'$.

\subsubsection{Leader Participant List}

A meeting leader maintains a ``leader participant list'' ($\LPL$) tabulating all the users in the meeting. For each user currently in the meeting, the $\LPL$ keeps track of a hash over their $\binding_i$, which includes their $IVK_i$, $pk_i$, $\userID$, and $\deviceid$, as well as their display name. For users who have left the meeting, the $\LPL$ tracks only their $\userID$, $\deviceid$, $\ivk_i$, and display name.

The $\LPL$ is used to drive the participant list in the user interface, which records both users
currently in the meeting and those who have left. It is possible that race conditions during leader
changes (caused by a bad network or a malicious server) could prevent some meeting participants from
being reflected in the ``left'' section of the participant list.

In Phase 1, the participant list only identifies users through their self-selected display name and
profile picture. Changes to participant's display names are relayed by the server to the leader, who
includes them in the $\LPL$, so a compromised server can change the display names of any meeting
participant. As such, display names should not be relied on to establish the participants'
identities. In the following phases, we will introduce a strong notion of identities to address
these limitations.

The $\LPL$ is represented as a sequence of operations such as adding a user to the meeting, or noting when a user has left. Every time there is such an operation, the leader increments a counter $v$ representing the total number of operations and signs over a data structure (called a \textit{link}) containing the counter, the hash of the previous link, and the current operation. If there are more than 20 links in the chain, the leader can coalesce all of the previous links into a smaller number of links. The old links are then deleted in order to save space.

Leaders post a signature over the latest link to the bulletin board whenever membership changes, and broadcast it over the signalling channel at designated ``heartbeat intervals'':
%
\begin{align*}
\zoomsignsign \left(\isk_{\LL}, \context, (\binding_{\LL} \| \mathsf{SHA256}(\LPL_v) \| v \| t \| \mkSequenceNumber ) \right)
\end{align*}
%
Where $\context$ is $\texttt{"Zoombase1-ClientOnly-Sig-Leader\-Participant\-List"}$, $t$ increments on every send, $v$ increments whenever the $\LPL$ changes, and $\mkSequenceNumber$ increments on every $\mk$ rotation.

By replaying the sequence of operations in the bulletin board, the other participants can reconstruct the current list of participants, so they know who to rekey for if the leader drops out and they become the new leader. Evil servers might try to withhold updates the leader makes here, to hide when bad actors are kicked out. As such, the leader also sends a low bandwidth ``heartbeat'' over the signalling channel. Heartbeats should go out at least every 10 seconds. All participants observe and verify these heartbeats, and if they fail to receive ten heartbeats in a row, they should drop out of the meeting.

When a leader does drop out of a meeting, the Zoom server picks a new leader arbitrarily and sends a signal to participants indicating that the leader has changed. The new leader then coalesces the chain as described above, and other participants verify that the new leader is present in the new $\LPL$.

For users in the meeting, clients remember the mapping between $(\userID, \deviceid)$ and hash over the corresponding $\binding_i$, and ensure that the hash remains stable across new links and leader changes. If a user leaves the meeting, it is enforced that any user who rejoins with the same $\userID$ and $\deviceid$ must have the same $\ivk_i$, but not necessarily the same $\pk_i$. These guarantees persist only over the course of a single meeting.

\subsubsection{Meeting Teardown}
At the end of the meeting, or when leaving a meeting early, all participants should discard all meeting keys, all keys derived from those meeting keys, and the ephemeral DH private keys $\sk_i$ they generated when they joined.

The intent here is to provide forward secrecy. That is, if an adversary can record all encrypted messages relayed between Zoom clients during the meeting, and can later recover all keys stored on a user's device after the meeting ends, they still cannot recover the meeting data.

\subsection{Meeting Leader Security Code}
If all participants can verify the authenticity of the leader's public key ($\ivk_{\LL}$), they are safe from ``meddler-in-the-middle attacks'' (MitM)\footnote{``Meddler-in-the-middle attack'' is also known as ``man-in-the-middle attack''.}. The Zoom client exposes the following ``meeting leader security code'' in the security tab:

$$\mathsf{Digits}(\SHATWO(\SHATWO(\texttt{"Zoombase-1-ClientOnly-MAC-SecurityCode"}) || \SHATWO(\ivk_{\LL})))$$

$\mathsf{Digits}$ extracts a string of 39 decimal digits from a SHA-256 hash, representing just over 129 bits of information. This representation is more human-readable and more internationalizable than the full hexadecimal hash. Crucially, every Zoom client in the meeting independently computes these codes from the $\ivk_{\LL}$ used in the handshake protocol. The length of the code is long enough to protect against second pre-image attacks. The leader reads out the meeting leader security code, and everyone in the meeting in turn does the same thing. If the code does not match, the participant should speak up in the meeting, and the leader should rotate the meeting key by kicking them out; they may be allowed to rejoin and try again.

If deep fake technology\footnote{We use ``deep fakes'' to refer to manipulated and/or fabricated audio/video data that uses synthetic media techniques to replace the likeness of one person with another. Using a ``deep fake'' (especially in real-time) of a meeting participant could potentially deceive others about the identity of that participant.} is a concern, or the participants do not know each other in advance, this verification can also happen over a different out-of-band secure channel.

Non-leader participants see a notification prompting them to re-perform the security code checks
whenever the meeting leader changes. These additional checks prevent a compromised Zoom server from
changing the meeting leader over the course of a meeting without being detected.

We considered other approaches to the meeting leader security code, such as mixing more of the handshake data into the displayed code. While more mixins would be more robust to attacks that try to confuse participants by mixing members from different meetings, we see a UX advantage of ``one leader, one code.''

\subsection{Changes to symmetric encryption}
Symmetric encryption in Zoom meetings will use AES-GCM with unique per-stream keys. As in the current design, all keys will be derived using a secure key derivation function (KDF) from a per-meeting Meeting Key ($\MK$). The meeting leader may rotate $\MK$ throughout the meeting.

All encrypted UDP packets are prefaced with the 4-byte $\mkSequenceNumber$, so participants know which version of $\MK$ to decrypt with.

\subsection{Security Properties}
\label{subsec:secprop}
The Identity Management System, Bulletin Board and Signaling Channel as enumerated in
Section~\ref{subsec:comp} are deployed by Zoom, and protect against Outsiders using TLS. Attackers
classified as Insiders by our threat model could monitor or meddle with these components. An Insider
monitoring such components (a passive attack) would expose meeting metadata, which is stated as a
limitation of the design in Section~\ref{subsec:limit}, but would not otherwise compromise the
confidentiality of the meeting.

We prevent Outsiders from joining meetings through passwords, waiting rooms and the other
non-cryptographic server-enforced access control features described in
Section~\ref{subsec:currdesign}. TLS and encryption of the meeting streams protect the
confidentiality and integrity of the meeting against Outsiders who might control the participants'
network.

Within the same meeting, the encrypted streams sent by each participants are protected against
replay attacks by using encryption nonces as counters. Even across different meetings, the streams
cannot be replayed thanks to the fact that participants delete all the ephemeral keys once a meeting
is over (which also guarantees forward secrecy). 

We aim to minimize the damage that an active Insider can perform without being detected. First, an
Insider can force participants to drop out of a meeting, as well as partition them into separate
meetings, each with its own leader, and add extra participants circumventing all the above
server-enforced features. Our protocol ensures that the meeting leader will be able to detect any
participant obtaining access to meeting encryption keys by monitoring the participant list for
unexpected entries (for example, by recognizing the participants' faces in their video streams).
Detecting unauthorized participants can be challenging in large meetings or in certain views, such
as when the leader is sharing their screen. In the following phases, a stronger notion of identity
will be leveraged to highlight potential eavesdroppers in the UI, such as those outside of the
host's organization or guest users.

A leader's view of the ``active'' participants list shows everyone who is in the meeting
(i.e., everyone who has been given current meeting key, or the next one if a key rotation is in
progress). If the host observes the same user leaving the meeting four or more times, the participant
list will reflect the number of observed leaves. This alert helps participants notice when an attacker repeatedly joins and leaves a meeting rapidly, which would allow the attacker to learn most of the meeting keys but remain in the ``left'' column for the majority of the meeting. Due to the distributed nature of the
system, this view of the participant list can be slightly out of date, both for the leader and for
the other participants, but will eventually converge if there is a long enough span without
participant churn. Some race conditions, such as participants quickly joining and leaving a meeting
while the meeting leader is changing, might also cause participants who left a meeting to not be
reflected in the ``left'' section of the participant list. This is necessary to guarantee a smooth
meeting experience under poor network conditions. Heartbeat messages ensure that a key is rotated in
a timely manner after an unwelcome participant has been removed from the meeting.

An active attacker can also try to perform a MitM attack against the meeting. This class of attacks
is mitigated by the ``Meeting Leader Security Code'', which should be re-checked every time a
participant joins (or re-joins) the meeting, or whenever the meeting leader security code changes as
indicated by a UI notification. 

\subsection{Local Key Security}
\label{subsec:lks}

Long-term keys are stored in the local operating system's keychain, but with some added protection
for two Zoom users using the same OS account.
%
Before storing keys in the local OS-provided keychain, we will encrypt them using a server-synchronized key-wrapping key.
%
When a user deprovisions their device, they will delete the keychain entry and the server will
delete its key. This also guarantees that keys cannot be recovered from a backup of a device that
has since been revoked.
%
If two users are using the same computer, the key-wrapping key prevents one user from being able to access the other's keys.

We use the committing AEAD scheme $\mathsf{CtE1}$~\cite{messagefranking} to prevent the server from supplying malicious key-wrapping keys.
%
On provisioning, after generating $\isk_i$, the client:
\begingroup
\RaggedRight
\begin{enumerate*}
\item Generates a 32-byte random string $\mathsf{KWK}$ and requests the server to store it persistently associated with the user.
\item Defines $\context \leftarrow \texttt{"Zoombase-1-ClientOnly-KDF-SecretStore"}$.
\item Computes $C \leftarrow \textsf{CtE1-Enc}(\textsf{K}{=}\textsf{KWK}, \textsf{H}{=}\context, \textsf{M}{=}\isk_i)$, where $\textsf{H}$ is the associated data parameter for the underlying AEAD, and stores it in the system keychain.
\end{enumerate*}
\endgroup

For \textsf{CtE1}, we will use $\textsf{HMAC}_{\textsf{SHA256}}$ as the
commitment function and \sodium{}~\cite{libsodium}'s \linebreak \texttt{crypto\_aead\_chacha20poly1305\_ietf} for AEAD.

We want to emphasize that this feature will not protect against an insider who also has access to a
user's device (in particular, by colluding with another user using the same device). It will also not
prevent different users of the same device from installing malware to steal the other user's keys.

\subsection{Abuse Management and Reporting}
If a user experiences abusive behavior and wishes to report it to Zoom's Trust and Safety team, they simply upload the unencrypted data normally collected in an abuse report (e.g. a description of the abuse and some portion of the meeting content) to Zoom for review. This protocol is imperfect, since it potentially could allow a bad meeting participant to ``frame'' an honest meeting participant for abuse that didn't happen. For the same reason, it allows an actual abuser to disavow uploaded evidence of their abuse. We think for now, the framing behavior is rare and only possible with access to good ``deep fake'' technology.

Future refinements are possible. Participants could sign their outgoing video streams, and other participants will only allow meetings to proceed if all streams are appropriately signed. This change would defeat the two attacks above, but with major drawbacks:
%
\begin{description}
    \item {\bf Performance:} Signing and verifying individual UDP video streams is expensive in terms of bandwidth and computation. More research is required to make this change practical.
    \item {\bf Repudiation:} Honest participants might not want an indelible record that they said something. They might understandably want to treat meetings as ephemeral in accordance with their standard data retention practices.
\end{description}

Given these challenges, we will revisit our decisions at a later date as we gain more operational experience with the current proposal.

\subsection{Guest Users}
Users can join video meetings in guest mode, in which they will generate fresh long-term keys for every meeting. This prevents other participants from tracing them across meetings by noticing when a long-term key is reused. Guest users include users who are not logged in.

\subsection{Areas to Improve in Phase I}
Phase I is geared toward a quick deployment and providing building blocks for future development phases. As such, there are some potential attacks we do not fully cover here:
\begin{description}
\item {\bf Meddler-in-the-Middle.} The ``meeting leader security code'' proposal above is a countermeasure for the classic ``meddler in the middle attack'', wherein Bob isn't actually connecting to Zoom, he's connecting to Eve who is proxying his communications. But our solution isn't perfect.  First off, it can be defeated by deep fake technology; second, it's clunky from a UX perspective; and third, the meeting leader may have to perform the ceremony multiple times with any users joining the meeting late. We could work toward addressing all of these concerns, but future phases of development provide better solutions by building strong notions of identity.
\item {\bf Anonymous Eavesdropper.} An adversary, in conjunction with a malicious Zoom server, types in a name of their choosing, turns off video, mutes their microphone and just observes. We'll fix this problem with better identity guarantees in Phase II.
\item {\bf Impersonation Attacks Within the Meeting.} Even if Alice and Bob are both authorized to be in the meeting, if Alice has the help of a malicious server, she can inject audio/video for Bob. Charlie would have no way of knowing that Bob's stream was being faked.
\end{description}

The upshot here is that in Phase II, we look to further strengthen meetings through better identity.
